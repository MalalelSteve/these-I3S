% !TEX root = ../sommaire.tex

\chapter{Introduction}

Cette classe \LaTeX\ est basée sur la classe \mytt{these-LINA} écrite par Frédéric Goualard. La classe a été modifiée pour correspondre au modèle de thèse demandé par l'ED STIC de Université Côte d'Azur.

Pour les autres ED, il suffit de supprimer le fichier \mytt{visuel.jpg} dans le dossier \mytt{img/} et de renommer le visuel correspondant à l'ED en \mytt{visuel.jpg}.

\section{Organisation}

  Pour compiler ce document vous pouvez utiliser le \mytt{Makefile}. Sinon, on m'a dit dans l'oreillette qu'il existait des éditeurs tel que Texmaker qui gèrent eux même la compilation.

  \subsection{Fichier principal et informations}
  Le fichier principal est le fichier \mytt{sommaire.tex}, la classe du document est \mytt{these-ISSS}.
  Différentes options sont possibles :
  \begin{itemize}
    \item \mytt{draft} pour une version brouillon du manuscrit, les liens sont désactivés et en noir, et les étiquettes des références sont affichées ;
    \item \mytt{revision} qui ajoute à chaque première page de chapitre la date et l'heure de la dernière compilation en pied de page ;
    \item \mytt{gray} pour une version en grayscale du manuscrit ;
    \item \mytt{final} (par défaut) pour la version finale du manuscrit ;
    \item \mytt{oneside} pour une impression recto et \mytt{twoside} (par défaut) pour une impression recto-verso ;
    \item \mytt{notitlepage} pour un manuscrit sans page de titre et \mytt{titlepage} (par défaut) avec page de titre.
  \end{itemize}

  Le titre du document ainsi que les informations correspondant au laboratoire, une co-tutelle ou les co-financeurs se trouvent dans le fichier \texttt{titreEtInfos.tex}.

  Les informations relatives au jury, au directeur, co-directeur et co-encadrant se trouvent dans le fichier \mytt{jury.tex}.

  Le résumée et les mots-clés en français et en anglais se trouvent dans le fichier \mytt{resume.tex}.

  \subsection{Les chapitres}
  À l'exception des remerciements et des notations, chaque chapitre a son dossier.
  Ce n'est pas nécessaire mais c'est plus facile pour s'y retrouver après, et dans chaque dossier il y a le fichier principal, et il peut y avoir d'autres fichiers ou dossiers (par exemple un dossier \mytt{img/}).

  Les chapitres sont ajoutés au document principal à l'aide des macros \verb|\input| ou \verb|\import|. À noter que la macro \verb|\include| ajoute forcément une nouvelle page et ne permet pas des inclusions en cascade, son utilisation est à éviter.

  \subsection{Les paquets}
  Les ajouts de paquets sont dans le fichier \mytt{mesmacros.sty}. Vous pouvez supprimer ou ajouter dans ce fichier d'autres paquets.

  Ce fichier utilise les fichiers suivant :
  \begin{itemize}
    \item \mytt{francisation.sty} dans lequel se trouvent les traductions des différents noms de section ou listes ;
    \item \mytt{couleurs.sty} dans lequel se trouvent des couleurs, et dans lequel vous pouvez ajouter des couleurs ;
    \item \mytt{theoremNames.sty} dans lequel se trouvent les définitions de théorèmes ;
    \item \mytt{theoremList.sty} dans lequel se trouve le code pour générer des listes de théorèmes, utilisé pour générer la liste des définitions et la liste des exemples à la fin de ce document ;
    \item \mytt{algo.sty} pour les algorithmes en français avec le package \mytt{algorithmic} et \mytt{colorationSyntaxique.sty} pour la coloration syntaxique des algorithmes avec le package \mytt{lstlistings}.
  \end{itemize}

  \subsection{PDF/A}

  Le document est désormais compilé pour êre compatible avec la norme PDF/A-3B. 
  Les modifications effectuées ont été faites à partir des instructions trouvées ici : \url{https://www.mathstat.dal.ca/~selinger/pdfa/}.

  Si cela ne vous convient pas vous pouvez commenter ou modifier les lignes 82 du fichier \mytt{mesmacros.sty} :

  \begin{framed}\vspace{-0.75em}
    \begin{verbatim}
\usepackage[a-3b]{pdfx}\end{verbatim}\vspace{-0.75em}
  \end{framed}

  Si vous voulez conserver cette norme veillez à modifier les informations contenues dans le fichier \mytt{sommaire.xmpdata}.

\section{Citations}

Dans ce document il y a 3 bibliographies :
\begin{itemize}
  \item la bibliographie pour les publications personnelles,
  \item la bibliographie pour les pages web,
  \item et la bibliographie générale.
\end{itemize}

Elles sont définies au début du fichier \mytt{sommaire.tex}.

\begin{framed}\vspace{-0.75em}
\begin{verbatim}
% Biblio pour les pages webs
\newcites{web}{Pages web}
% Biblio pour mes publications
\newcites{mine}{Mes publications}
\end{verbatim}\vspace{-0.75em}
\end{framed}

On utilise pour cela le package \mytt{multibib}, vous pouvez en ajouter ou renommer celles existantes, il ne faut pas oublier de modifier le fichier \mytt{Makefile} pour compiler les nouvelles bibliographies ou supprimer la compilation de celles que vous n'utilisez pas.

Par défaut, les bibliographies utilisent le package qui cite correctement des pages web.
Le package impose le style \mytt{apacite} qui utilise un système auteur-date.
Pour changer de style de la bibliographie, vous devrez probablement supprimer le package.

\paragraph{Les publications dans la bibliographie générale} doivent être citées avec la macro \mytt{\textbackslash cite\{key\}}.

\begin{framed}
\noindent\cite{T2012}\vspace{-0.75em}
\begin{verbatim}\cite{T2012}\end{verbatim}\vspace{-0.75em}
\end{framed}

\paragraph{Les publications pour la section pages web} doivent être citées avec la macro \mytt{\textbackslash citeweb\{key\}}.

\begin{framed}
\noindent\citeweb{AC9} \vspace{-0.75em}
\begin{verbatim}\citeweb{AC9}\end{verbatim}\vspace{-0.75em}
\end{framed}

\paragraph{Les publications pour la section correspondant à vos publications} doivent être citées avec la macro \mytt{\textbackslash citemine\{key\}}.

\begin{framed}
\noindent\citemine{DHHS2011} \vspace{-0.75em}
\begin{verbatim}\citemine{DHHS2011}\end{verbatim}\vspace{-0.75em}
\end{framed}

Pour référencer vos publications par la suite, vous pouvez utiliser la macro \mytt{\textbackslash cite} ou la macro \mytt{\textbackslash citemine}.
Si vous souhaitez que votre publication apparaisse dans la bibliographie générale à la fin du manuscript vous pouvez soit la citer avec \mytt{\textbackslash cite} ou l'insérer avec \mytt{\textbackslash nocite}. \nocite{DHHS2011}

Attention si vous utilisez un style de bibliographie numérotant les références, vous aurez des problèmes de numérotation si une publication est à la fois citée en utilisant la macro \mytt{\textbackslash cite} et la macro \mytt{\textbackslash citemine}.



Chaque bibliographie peut ensuite être affichée en utilisant les lignes suivantes :

\begin{framed}\vspace{-0.75em}
\begin{verbatim}\bibliographystyle<s>{apacite}
\bibliography<s>{biblio}
\end{verbatim}\vspace{-0.75em}
\end{framed}

où \verb|<s>| correspond à la bibliographie que l'on veut afficher.

Pour redéfinir l'affichage, il suffit de redéfinir le \verb|\bibtitle|, pour que ce soit un chapitre utilisez la commande suivante :

\begin{framed}\vspace{-0.75em}
\begin{verbatim}\renewcommand{\bibtitle}{\bibchapter}
\end{verbatim}\vspace{-0.75em}
\end{framed}

Pour que ça apparaisse comme une section, utilisez la commande suivante :

\begin{framed}\vspace{-0.75em}
\begin{verbatim}\renewcommand{\bibtitle}{\section*{\refname}}
\end{verbatim}\vspace{-0.75em}
\end{framed}


\section{Exemples d'utilisation}

Des exemples d'utilisation des différents paquets sont présents dans ce document.


Des citations pour compléter les bibliographies \citemine{HHSD2011,HHSD2012}, \citeweb{AC9}.


Des citations pour compléter les bibliographies \citeA{HHSD2011}.

\citeAweb{AC9}.

Une référence citée avec \mytt{\textbackslash citemine} \citemine{DHHS2011} et \mytt{\textbackslash cite} \cite{DHHS2011} pour voir.


\renewcommand{\bibtitle}{\section*{\refname}}
\bibliographystylemine{apacite}
\bibliographymine{biblio}
