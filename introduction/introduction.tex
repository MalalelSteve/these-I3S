% !TEX root = ../sommaire.tex

\chapter{Introduction}

Cette classe \LaTeX est basée sur la classe \texttt{these-LINA} écrite par Frédéric Goualard. La classe a été modifiée pour correspondre au modèle de thèse demandé par l'ED STIC de l'Université Côte d'Azur.

Pour les autres ED, il suffit de supprimer le fichier \texttt{visuel.jpg} dans le dossier \texttt{img/} et de renommer le visuel correspondant à l'ED en \texttt{visuel.jpg}.

\section{Organisation}
  
  Pour compiler ce document vous pouvez utiliser le \texttt{Makefile}.

  \subsection{Fichier principal et informations}
  Le fichier principal est le fichier \texttt{sommaire.tex}, la classe du document est \texttt{these-ISSS}. 
  Différentes options sont possibles :
  \begin{itemize}
    \item \texttt{draft} pour une version brouillon du manuscrit, les liens sont désactivés et en noir, et les étiquettes des références sont affichées ;
    \item \texttt{revision} qui ajoute à chaque première page de chapitre la date et l'heure de la dernière compilation en pied de page ;
    \item \texttt{gray} pour une version en grayscale du manuscrit ;
    \item \texttt{final} (par défaut) pour la version finale du manuscrit ;
    \item \texttt{oneside} pour une impression recto et \texttt{twoside} (par défaut) pour une impression recto-verso ;
    \item \texttt{notitlepage} pour un manuscrit sans page de titre et \texttt{titlepage} (par défaut) avec page de titre.
  \end{itemize}
  
  Le titre du document ainsi que les informations correspondant au laboratoire, une co-tutelle ou les co-financeurs se trouvent dans le fichier \texttt{titreEtInfos.tex}.
  
  Les informations relatives au jury, au directeur, co-directeur et co-encadrant se trouvent dans le fichier \texttt{jury.tex}.
  
  Le résumée et les mots-clés en français et en anglais se trouvent dans le fichier \texttt{resume.tex}.
  
  \subsection{Les chapitres}
  À l'exception des remerciements et des notations, chaque chapitre a son dossier. 
  Ce n'est pas nécessaire mais c'est plus facile pour s'y retrouver après, et dans chaque dossier il y a le fichier principal, et il peut y avoir d'autre fichiers ou dossiers (par exemple un dossier \texttt{img/}).
  
  Les chapitres sont ajoutés aux document principal à l'aide des macros \verb|\input| ou \verb|\import|. À noter que la macro \verb|\include| ajoute forcément une nouvelle page et ne permet pas des inclusions en cascades, son utilisation est à éviter.
  
  \subsection{Les paquets}
  Les ajouts de paquets sont dans le fichier \texttt{mesmacros.sty}. Vous pouvez supprimer ou ajouter dans ce fichier d'autres paquets. 
  
  Ce fichier utilise les fichiers suivant :
  \begin{itemize}
    \item \texttt{francisation.sty} dans lequel se trouvent les traductions des différents noms de section ou listes ;
    \item \texttt{couleurs.sty} dans lequel se trouvent des couleurs, et dans lequel vous pouvez ajouter des couleurs ;
    \item \texttt{theoremNames.sty} dans lequel se trouvent les définitions de théorèmes ;
    \item \texttt{theoremList.sty} dans lequel se trouve le code pour générer des listes de théorèmes, utilisé pour générer la liste des définitions et la liste des exemples à la fin de ce document ;
    \item \texttt{algo.sty} pour les algorithmes en français avec le package \texttt{algorithmic} et \texttt{colorationSyntaxique.sty} pour la coloration syntaxique des algorithmes avec le package \texttt{lstlistings}.
  \end{itemize}

\section{Citations}

Dans ce document on a créé 3 bibliographie :
\begin{itemize}
  \item la bibliographie pour les publications personnelles,
  \item la bibliographie pour les pages web,
  \item et la bibliographie générale.
\end{itemize}

Elles sont définies au début du fichier \texttt{sommaire.tex}.

\begin{framed}
\begin{verbatim}
% Biblio pour les pages webs
\newcites{web}{Pages web}
% Biblio pour mes publications
\newcites{mine}{Mes publications}
\end{verbatim}\vspace{-0.5em}
\end{framed}

On utilise pour cela le package \texttt{multibib}, vous pouvez ajouter les votre ou renommer celles existantes, il ne faut pas oublier de modifier le fichier \texttt{Makefile} pour compiler les nouvelles bibliographies ou supprimer la compilation de celles que vous n'utiliser pas.

\paragraph{Les publications dans la bibliographie générale} doivent être citées avec la macro \texttt{\textbackslash cite\{key\}}.

\begin{framed}
\noindent\cite{T2012}\vspace{-0.5em}
\begin{verbatim}\cite{T2012}\end{verbatim}\vspace{-0.75em}
\end{framed}

\paragraph{Les publications pour la section pages web} doivent être citées avec la macro \texttt{\textbackslash citeweb\{key\}}.

\begin{framed}
\noindent\citeweb{AC9} \vspace{-0.5em}
\begin{verbatim}\citeweb{AC9}\end{verbatim}\vspace{-0.75em}
\end{framed}

\paragraph{Les publications pour la section correspondant à vos publications} doivent être citées avec la macro \texttt{\textbackslash citemine\{key\}}.

\begin{framed}
\noindent\citemine{DHHS2011} \vspace{-0.5em}
\begin{verbatim}\citemine{DHHS2011}\end{verbatim}\vspace{-0.75em}
\end{framed}

Pour référencer vos publications par la suite, vous pouvez utiliser la macro \texttt{\textbackslash cite} ou la macro \texttt{\textbackslash citemine}.
Si vous souhaitez que votre publication apparaisse dans la bibliographie générale à la fin du manuscript vous pouvez soit la citer avec \texttt{\textbackslash cite} ou l'insérer avec \texttt{\textbackslash nocite}. \nocite{DHHS2011} 

Attention si vous utilisez un style de bibliographie numérotant les références, vous aurez des problèmes de numérotation si une publication est à la fois citée en utilisant la macro \texttt{\textbackslash cite} et la macro \texttt{\textbackslash citemine}.


Chaque bibliographie peut ensuite être affichée en utilisant les lignes suivantes :

\begin{framed}
\begin{verbatim}\bibliographystyle<s>{apalike}
\bibliography<s>{biblio}
\end{verbatim}\vspace{-0.5em}
\end{framed}

où \verb|<s>| correspond à la bibliographie que l'on veut afficher.


\section{Exemples d'utilisation}

Des exemples d'utilisation des différents paquets sont présents dans ce document.


Des citations pour compléter les bibliographies \citemine{HHSD2011,HHSD2012}, \citeweb{AC9}.
 

  
\renewcommand{\bibtitle}{\section*{\refname}}
\bibliographystylemine{apalike}
\bibliographymine{biblio}


