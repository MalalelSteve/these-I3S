% !TEX root = ../sommaire.tex

\chapter{Chapitre 5}

% !TEX root = ../sommaire.tex

\chapintro{En fait, j'en ai marre de mettre des résumés en anglais.}

\tikzstyle{nodedefault} = [draw, circle]
\tikzstyle{edgedefault} = [draw]

\tikzstyle{n1} = [nodedefault]
\tikzstyle{n2} = [nodedefault]
\tikzstyle{n3} = [nodedefault]
\tikzstyle{n4} = [nodedefault]
\tikzstyle{n5} = [nodedefault]
\tikzstyle{n6} = [nodedefault]
\tikzstyle{n7} = [nodedefault]
\tikzstyle{n8} = [nodedefault]
\tikzstyle{n9} = [nodedefault]

\tikzstyle{a12} = [edgedefault]
\tikzstyle{a13} = [edgedefault]
\tikzstyle{a25} = [edgedefault]
\tikzstyle{a34} = [edgedefault]
\tikzstyle{a46} = [edgedefault]
\tikzstyle{a56} = [edgedefault]
\tikzstyle{a67} = [edgedefault]
\tikzstyle{a68} = [edgedefault]
\tikzstyle{a79} = [edgedefault]
\tikzstyle{a89} = [edgedefault]

% style pour les noeuds roses
\tikzstyle{nrose} = [pink]

% style pour les noeuds bleus
\tikzstyle{nbleu} = [white, fill = bleu]

%%%%%%%%%%%%%%%%%%%%%%%%%%%%%
\section{Section 1}
%%%%%%%%%%%%%%%%%%%%%%%%%%%%%

\begin{figure}
  \centering
    
  \begin{tikzpicture}[every node/.style = {nodedefault}, every path/.style = {edgedefault}] 
    
%%% Définition du graphe %%%

%% Les noeuds
\node[n1] (1) at (0, 0) {NCE};
\node[n2, above right = of 1] (2) {CDG};
\node[n3, below right = of 1] (3) {YUL};
\node[n4, right = of 2] (4) {4};
\node[n5, right = of 3] (5) {5};
\node[n6, below right = of 4] (6) {6};
\node[n7, above right = of 6] (7) {7};
\node[n8, below right = of 6] (8) {8};
\node[n9, below right = of 7] (9) {9};

%% Les arêtes
\path[a12] (1) -- (2);
\path[a13] (1) -- (3);
\path[a25] (2) -- (5);
\path[a34] (3) -- (4);
\path[a46] (4) -- (6);
\path[a56] (5) -- (6);
\path[a67] (6) -- (7);
\path[a68] (6) -- (8);
\path[a79] (7) -- (9);
\path[a89] (8) -- (9);
  \end{tikzpicture}
    
  \caption{Graphe de départ}
  \label{fig:graphe}
\end{figure}

La figure \ref{fig:graphe} est vraiment magnifique.

%%%%%%%%%%%%%%%%%%%%%%%%%%%%%
\section{Section 2}
%%%%%%%%%%%%%%%%%%%%%%%%%%%%%

  \begin{figure}
    \centering
    
    \begin{tikzpicture}[every node/.style = {nodedefault}, every path/.style = {edgedefault}]
      % redéfinitions des nœuds en rose
      \tikzset{n1/.style = nrose}
      \tikzset{n9/.style = nrose}
      
      % et en rectangle
      \tikzset{n2/.style = {rectangle, pink}}
      
      % redéfinitions de nœuds bleus
      \tikzset{n6/.style = nbleu}
      
      % redéfinitions des arêtes
      \tikzset{a12/.style = {->, vert, thick}}
      \tikzset{a25/.style = {-), vert, thick}}
      \tikzset{a56/.style = {-], vert, thick}}
      \tikzset{a67/.style = {(-], vert, thick}}
      \tikzset{a79/.style = {o-[, vert, thick}}
      
      \tikzset{a13/.style = {red, dashed, thick}}
      
      
%%% Définition du graphe %%%

%% Les noeuds
\node[n1] (1) at (0, 0) {NCE};
\node[n2, above right = of 1] (2) {CDG};
\node[n3, below right = of 1] (3) {YUL};
\node[n4, right = of 2] (4) {4};
\node[n5, right = of 3] (5) {5};
\node[n6, below right = of 4] (6) {6};
\node[n7, above right = of 6] (7) {7};
\node[n8, below right = of 6] (8) {8};
\node[n9, below right = of 7] (9) {9};

%% Les arêtes
\path[a12] (1) -- (2);
\path[a13] (1) -- (3);
\path[a25] (2) -- (5);
\path[a34] (3) -- (4);
\path[a46] (4) -- (6);
\path[a56] (5) -- (6);
\path[a67] (6) -- (7);
\path[a68] (6) -- (8);
\path[a79] (7) -- (9);
\path[a89] (8) -- (9);      
    \end{tikzpicture}
    
    \caption{Graphe avec des nœuds en rose et d'autres en bleu}
  \label{fig:grapherb}
  \end{figure}

%%%%%%%%%%%%%%%%%%%%%%%%%%%%%
\section{Section 3}

  \begin{figure}
    \centering
    
    \begin{tikzpicture}[every node/.style = {nodedefault}, every path/.style = {edgedefault}]
      
%%% Définition du graphe %%%

%% Les noeuds
\node[n1] (1) at (0, 0) {NCE};
\node[n2, above right = of 1] (2) {CDG};
\node[n3, below right = of 1] (3) {YUL};
\node[n4, right = of 2] (4) {4};
\node[n5, right = of 3] (5) {5};
\node[n6, below right = of 4] (6) {6};
\node[n7, above right = of 6] (7) {7};
\node[n8, below right = of 6] (8) {8};
\node[n9, below right = of 7] (9) {9};

%% Les arêtes
\path[a12] (1) -- (2);
\path[a13] (1) -- (3);
\path[a25] (2) -- (5);
\path[a34] (3) -- (4);
\path[a46] (4) -- (6);
\path[a56] (5) -- (6);
\path[a67] (6) -- (7);
\path[a68] (6) -- (8);
\path[a79] (7) -- (9);
\path[a89] (8) -- (9);
      
      \tikzset{every node/.style = {}} % on ne veut plus de cercle autour des nœuds
      \node[red] () at ($(1) !.5! (2)$) {toto}; % directement au milieu
      
      \node[bleu, right] () at ($(4) !.35! (6)$) {tata}; % à droite, à 0.35 de (4)
      
      \tikzset{every path/.style = {}} % on ne veut plus dessiner les chemins
      \path (7) -- (9) node [midway, above, sloped, vert] () {titi};
      
      \path (8) -- (9) node [midway, below, sloped, pink] (ici) {ici};
      
      \node[right = 0.5cm, pink] (regardez) at (9 |- 8) {Regardez};  % le nœud se trouve à 0.5cm à droite du point ayant la même abscisse que (9) et la même ordonnée que (8);
      
      \draw[->, pink, very thick] (regardez) -- (ici);

    \end{tikzpicture}
    
    \caption{Graphe de départ avec une étiquette sur certaines arêtes}
  \label{fig:grapheetiq}
  \end{figure}
%%%%%%%%%%%%%%%%%%%%%%%%%%%%%

%%%%%%%%%%%%%%%%%%%%%%%%%%%%%
\section{Conclusion}
%%%%%%%%%%%%%%%%%%%%%%%%%%%%%